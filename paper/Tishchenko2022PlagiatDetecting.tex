\documentclass[12pt, twoside]{article}
\usepackage{jmlda}
\usepackage[]{algorithmic}
\usepackage{graphicx}
\usepackage{multicol}
\usepackage{caption}
\usepackage{subfig}
\newcommand{\hdir}{.}
\newtheorem{statement}{Утверждение}

\begin{document}

\title
    []
    {Кроссязычный поиск дубликатов}

\author
    [Е.\,В.~Тищенко]
    {Е.\,В.~Тищенко, К.\,В.~Воронцов}
    [Е.\,В.~Тищенко$^1$, К.\,В.~Воронцов$^2$]

\abstract{
В данной статье рассматривается задача кроссязычного поиска текстового плагиата. Современные методы векторизации документов и последующего поиска совпадений преимущественно основываются на одном языке, что приводит к возникновению проблемы возникновения однообразных мультиязыковых коллекций документов.

Целью работы является получение модели, кодирующей информацию о распределение слов в тексте независимо от их языковой принадлежности, при этом ограниченной по размеру и времени обучения для ее практического использования.
}

\maketitle
\linenumbers
\section{Введение}
...

\end{document}