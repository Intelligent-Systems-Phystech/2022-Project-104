\documentclass[12pt, twoside]{article}
\usepackage{jmlda}
\usepackage[]{algorithmic}
\usepackage{graphicx}
\usepackage{multicol}
\usepackage{caption}
\usepackage{subfig}
\newcommand{\hdir}{.}
\newtheorem{statement}{Утверждение}

\begin{document}

\title
    []
    {Кроссязычный поиск дубликатов}

\author
    [Е.\,В.~Тищенко]
    {Е.\,В.~Тищенко, К.\,В.~Воронцов}
    [Е.\,В.~Тищенко$^1$, К.\,В.~Воронцов$^2$, П.\,С.~Потапова$^3$]

\abstract{
В данной статье рассматривается задача кроссязычного поиска текстового плагиата. Современные методы векторизации документов и поиска совпадений преимущественно основываются на одном языке, что приводит к проблеме возникновения однообразных мультиязыковых коллекций документов.

Целью работы является получение модели, выделяющей информацию о распределении слов в тексте независимо от их языковой принадлежности, при этом ограниченной по размеру и времени обучения для ее практического использования.
}

\maketitle
\linenumbers
\section{Введение}

Задача поиска и распознавания плагиата является особенно важной в эпоху информационных технологий. Невозможно проверить все документы общей тематики на плагиат в силу размеров глобальной сети. Задача становится особенно трудной, так как возникает множество документов, являющихся переводом исходных работ. Согласно исследованиям Donald L. McCabe,\cite{donaldSurvey}  $ 36\% $ студентов американских университетов перефразировали или копировали информацию из всемирной паутины без ссылки на источник. 

Современные методы векторизации документов для поиска плагиата  \cite{methodMLPlag, regression} преимущественно ограничиваются одним языком. В таком случае возникает проблема создания единообразной системы получения векторных эмбедингов мультиязыковой коллекции документов. Для возможности применения модели за пределами научных экспериментов ставятся технические ограничения ресурсами сервера, а именно на размер модели, временную сложность обучения.

Объектом исследования являются мультиязыковые тематические модели, алгоритмы поиска документов в текстовой коллекции по словам и документам, способы векторного представления слов и документов запроса и коллекции, применяемые для поиска дубликатов независимо от языковой принадлежности.

Тематическая модель коллекции текстовых документов определяет, к каким темам относится каждый документ и какие слова или термины образуют каждую тему. Вероятностная тематическая модель описывает каждую тему дискретным распределением вероятностей слов, а каждый документ $\--$ дискретным распределением вероятностей тем. Тематическая модель преобразует любой текст в вектор вероятностей тем. 

Для решения задачи была построена мультимодальная тематическая модель. Такая тематическая модель описывает документы, содержащие метаданные наряду с основым текстом. Метаданные позволяют более точно определять тематику документа. В качестве модальностей использовались 100 языков, а также научные рубрики. Использование языков в качестве модальностей позволяет получить векторное представление текста, независимое от оригинального языка текста, что позволяет решать проблему поиска плагиата без ограничения на язык статьи. Во время предобработки текста используется BPE токенизация $\--$ итеративная замена наиболее встречаемой пары символов на символ, который не встречается в слове. Это позволяет существенно уменьшить объем изначального словаря для практического применения модели.

Целью эксперимента является построение модели, исследование влияния регуляризации и предобработки текстовых данных на качество поиска, подбор разнообразных функций для сравнения тематических расстояних векторов а также поиск эвристик для улучшения точности предсказаний модели. В качестве обучающих данных используются статьи с сайта Wikipedia, а также выборка научных статей из научной электронной библиотеки eLIBRARY.ru.

\section{Постановка задачи}

Пусть D $\--$ некоторая коллекция документов. Кандидатом на дубликат  для документа $d \in D$ обозначим такой элемент коллекции f(d), что  $$f(d) = \argmin_{d' \in D \setminus d} distance(m(d), m(d'))$$ 
В качестве функции $distance$ может быть использована произвольная метрика векторного пространства. Функция  $m(d) \--$ модель, осуществляющая преобразование документа в векторное пространство. Качество модели поиска дубликатов измеряется на тестовой выборке при помощи двух метрик сопоставления переводов:

1. Средняя частота, с которой документ-запрос попадает в топ 10\%

2. Средний процент документов в топ 10\% документов-переводов, которые имеют такую же рубрику, что у документа-запроса

Требуется построить тематическую модель, позволяющую получить векторное представление документов, при этом ошибка по метрикам качества должна превосходить 0.9 и 0.3 соответственно.


\begin{thebibliography}{3}

\bibitem{donaldSurvey}
    \BibAuthor{Donald L. McCabe}
    Cheating among college and university students: A North American perspective//
    \BibJournal{International Journal for Educational Integrity}, 2005
    
\bibitem{methodMLPlag}
    \BibAuthor{Zdenek Ceska, Michal Toman, and Karel Jezek}
    Multilingual Plagiarism Detection//
    \BibJournal{Artificial Intelligence: Methodology, Systems and Applications}, 2008
    
\bibitem{regression}
    \BibAuthor{Duygu Ataman, Jose G. C. de Souza, Marco Turchi, Matteo Negri}
    Cross-lingual Semantic Similarity Measurement Using Quality Estimation Features and Compositional Bilingual Word Embeddings//

	  
\end{thebibliography}

\end{document}